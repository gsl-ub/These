% --- Langue et encodage (gérés par fontspec si LuaLaTeX/XeLaTeX) ---
\usepackage[english, french]{babel}         % Gestion multilingue, langue française par défaut (dernière) : typographie, noms automatiques, utiliser \selectlanguage{english} pour basculer en anglais et \begin{otherlanguage}{french} pour des inclusions fr suivi de \end{otherlanguage} puis de \selectlanguage{english} pour reprendre une partie en anglais.
\usepackage{csquotes}                       % Guillemets typographiques adaptés à la langue
% A supprimer lorsque j'utiliserai des polices personnalisées.
%\usepackage[utf8]{inputenc}                 % Encodage UTF-8 (utile en local)
\usepackage[T1]{fontenc}                    % Sortie correcte des accents
\usepackage{lmodern}                        % Police vectorielle moderne

% --- Mathématiques et symboles scientifiques ---
\usepackage{amsmath, amssymb, amsfonts}     % Pour les formules et symboles mathématiques
\usepackage{siunitx}                        % Gestion des unités scientifiques (1.23 en et 1,23 fr) \SI{3.5}{mg/L} -> possibilité aussi de faire des unités personnalisées voir chatGPT au besoin

% --- Images et figures ---
\usepackage{graphicx}                       % Insertion d'images
\usepackage{float}                          % Placement forcé avec [H]
\usepackage[font=small,labelfont=bf,width=0.8\textwidth]{caption}   % Style des légendes
\usepackage{subcaption}                     % Sous-figures
% \usepackage{svg}                          % Pour inclure des fichiers SVG

% --- Tableaux et couleurs ---
\usepackage[table,xcdraw]{xcolor}           % Couleurs dans les tableaux
\usepackage{multirow}                       % Fusion de cellules
% \usepackage{booktabs}                     % Tableaux élégants si besoin (\toprule...)

% --- Bibliographie ---
\usepackage[backend=biber, style=apa, sorting=nyt, url=false]{biblatex}
\addbibresource{references.bib}             % Fichier de bibliographie
% \addbibresource{biblio_manuelle.bib}      % Autre fichier possible -> utiliser betterbiblatex... ?
% Réglages typographiques utiles
\setcounter{biburllcpenalty}{7000}
\setcounter{biburlucpenalty}{8000}
%Augmente le seuil de coupure dans les URL en bibliographie. Cela évite les césures moches dans les longues URL.
%\renewcommand*{\labelnamepunct}{\newunitpunct\par}% Met un saut de ligne après les noms d’auteur au lieu d’un point, pour une bibliographie plus aérée.
%\renewbibmacro{in:}{\newline In:}           % Met également un saut de ligne avant le mot In: (chapitre dans un ouvrage), pour un rendu plus clair.
%\DefineBibliographyExtras{french}{\restorecommand\mkbibnamefamily} % Restaure un comportement cohérent en français pour la bibliographie dans la mise en forme des noms de famille

% --- Glossaire et acronymes ---
\usepackage[toc, acronym]{glossaries}
\makeglossaries
% \renewcommand*{\glsentryfmt}[1]{#1}       % Désactiver l'insertion automatique de la définition dans le texte

% --- Hyperliens ---
\usepackage{hyperref}
% \usepackage{cleveref}                     % Références intelligentes (ex: \cref{fig:truc})

% --- Commentaires en marge (optionnels) ---
\usepackage[textsize=footnotesize, color=orange!20]{todonotes}

% --- Gestion des annexes ---
%\usepackage{appendix}                       % Gestion propre des annexes : inutile avec la classe memoir