% Utilisation de polices système/personnalisées avec fontspec
% Pour utiliser fontspec (nécessite xelatex ou lualatex) -> Xualatex mieux
\usepackage{fontspec}

% Police principale (corps du texte)
\setmainfont{RobotoSerif}[% Le nom exact peut dépendre du système, vérifiez les fichiers .ttf/.otf
  Path=Fonts/Roboto_Serif/static/,
  UprightFont={*_72pt-Light},    % Pour le texte normal (Light 300)
  ItalicFont={*-LightItalic}, % Pour l'italique (Light 300 Italic)
  BoldFont={*-SemiBold},        % Pour le gras (Regular)
  BoldItalicFont={*-SemiBoldItalic},    % Pour le gras italique (Italic)
  Extension=.ttf
]

% Police pour les titres : Roboto Slab Light 300
\newfontfamily\titrefont[
    Path=Fonts/Roboto_Slab/static/,
    UprightFont = *-Light.ttf,
    BoldFont = *-Regular.ttf
]{RobotoSlab}

% Couleurs globales du document : éclaircir les polices au besoin
%\definecolor{darkgraytext}{gray}{0.15}
%\color{darkgraytext}

% Polices des titres de section (Roboto Slab Light 300)
% memoir utilise des commandes spécifiques pour les polices des titres de section
%\renewcommand{\chapnamefont}{\normalfont\fontspec{Roboto Slab}[UprightFont={*-Light300}]} % Nom du chapitre (si affiché)
%\renewcommand{\chapnumfont}{\normalfont\fontspec{Roboto Slab}[UprightFont={*-Light300}]} % Numéro de chapitre (si affiché)
%\renewcommand{\chaptitlefont}{\normalfont\fontspec{Roboto Slab}[UprightFont={*-Light300}]\Huge} % Titre de chapitre
%\renewcommand{\sectfont}{\normalfont\fontspec{Roboto Slab}[UprightFont={*-Light300}]\Large} % Titre de section
%\renewcommand{\subsectfont}{\normalfont\fontspec{Roboto Slab}[UprightFont={*-Light300}]\large} % Titre de sous-section
%\renewcommand{\subsubsectfont}{\normalfont\fontspec{Roboto Slab}[UprightFont={*-Light300}]\normalsize} % Titre de sous-sous-section
% Et ainsi de suite pour les autres niveaux si vous les utilisez (paragraph, subparagraph)
%\renewcommand{\chaptitlefont}{\titlefont\LARGE} % Titre
% tailles polices : \tiny < \scriptsize < \footnotesize < \small < \normalsize < \large < \Large < \LARGE < \huge < \Huge
% graisse : \bfseries

% Ne pas oublier de recaluler la largeur des blocs texte après un changement de police, de graisse de police ou même de taille !
% Bien tout définir...