% Marges asymétriques pour impression recto-verso
\usepackage{geometry}
\geometry{
  top=2.5cm,
  bottom=2.5cm,
  inner=2.8cm,
  outer=3.5cm,
  bindingoffset=1cm,
  includehead % Permet de conserver un espacement de 2.5 par rapport aux en-têtes plutôt que les bords de page.
}

% Interligne plus aéré
%\usepackage{setspace}   % Utilisation de \setstretch{1.05} pour la gestion des interlignes.
%\setstretch{1.05} % Reglage des interlignes
\linespread{1.05}   % Memoir intègre en interne la gestion de l'interlignage...

% Paragraphe : pas d'indentation + saut entre paragraphes
\setlength{\parindent}{0pt}
\setlength{\parskip}{0.7em}

% Améliorations typographiques
\usepackage{microtype} % césure fine, justification plus jolie

% Style des titres (sobre) % maîtrise fine
%\usepackage{titlesec}
%\titleformat{\chapter}[hang]{\Large\bfseries\titlefont}{}{0pt}{}
%\titleformat{\section}[hang]{\large\bfseries}{}{0pt}{}
%\titleformat{\subsection}[runin]{\bfseries}{}{0pt}{}
%\titlespacing*{\chapter}{0pt}{20pt}{10pt}
%\titlespacing*{\section}{0pt}{12pt}{6pt}

% Style des titres natif à memoir (simple et stable)
%\chapterstyle{section} % ou 'default', 'dash', 'hangnum', etc.

% Appliquer la bonne police (définie comme \titlefont dans fonts.tex) aux titres de chapitres
%\renewcommand{\chapnamefont}{\titlefont\LARGE}  % "Chapitre"
%\renewcommand{\chapnumfont}{\titlefont\LARGE}   % Numéro
%\renewcommand{\chaptitlefont}{\titlefont\LARGE} % Titre
% tailles polices : \tiny < \scriptsize < \footnotesize < \small < \normalsize < \large < \Large < \LARGE < \huge < \Huge
% graisse : \bfseries

% Numérotation continue des sections (optionnel)
% \counterwithout{section}{chapter} % décommente si tu ne veux pas 1.1, 1.2, mais juste 1, 2, 3

% En-têtes et pieds de page classiques (peu chargés)
\pagestyle{plain} % pas d'en-tête et numerotation en bas centrée.
% Possibilité avec le package fancyhdr de maîtriser les entêtes... plus tard.

% ============================
% Mise en page - OLD
% ============================
%\usepackage[top=1.5cm,bottom=2cm,left=2.5cm,right=2.5cm,includehead]{geometry} % Marges personnalisées
%\usepackage{setspace}                       % Pour gérer l'interligne (\onehalfspacing, etc.)
% \usepackage{pdfpages}                     % Pour gérer l'insertion de PDF...
% \usepackage{afterpage}                    % Pour pouvoir forcer les sauts de page proprement : \afterpage{\clearpage} force l’insertion de toutes les figures en attente juste après cette page
% \usepackage[compact]{titlesec}            % Mise en forme des titres plus compacte
% \usepackage[table,xcdraw]{xcolor}         % Gérer les couleurs dans le document notamment dans les tableaux
% voir pour utiliser police / font perso (ex: Lato).