\usepackage[pass]{geometry} % Pour des mises en forme local dans mon doc sans prise de tête liminaires et articles
% ---- Utilisation des fonctionnalités de la classe memoir ----
% 1. Taille du papier (définie lors de l'appel à la class memoir
%\settrims{0pt}{0pt} %\settrims{<haut>}{<bord_de_coupe>} en cas de bords perdus, en définir 2

% 2. Bloc de texte & marges
%\settypeblocksize{<hauteur>}{<largeur>}{<ratio>}
%\setlrmargins{⟨spine⟩}{⟨edge⟩}{⟨ratio⟩}
%\setulmargins{<haut>}{<bas>}{<ratio>} % Utilisez cette commande si la hauteur de votre bloc de texte est déjà fixe (par \settypeblocksize) et que vous positionnez le bloc en ajustant les marges, ex : \settypeblocksize{41\baselineskip + \topskip}{33pc}{*}
\setlrmarginsandblock{2.5cm}{5cm}{*} % Marges latérales (gauche/droite)
\setulmarginsandblock{3cm}{3cm}{*} % Marges haut/bas

% 3. En-têtes et pieds de page
\setheadfoot{\onelineskip}{2\onelineskip} % \setheadfoot{<hauteur_entete>}{<espacement_pied_de_page>}
%\setheaderspaces{1.5\onelineskip}{*} % \setheaderspaces{<chute_entete>}{<separation_entete_texte>}{<ratio>}
\setlength{\headheight}{15pt}

% 4. Notes marginales
\setmarginnotes{1em}{3.5cm}{1.5em} %\setmarginnotes{<separation>}{<largeur>}{<poussee>}

% Interligne plus aéré
\linespread{1.05}   % Memoir intègre en interne la gestion de l'interlignage...

% Paragraphe : pas d'indentation + saut entre paragraphes
\setlength{\parindent}{1em} % inutile
\setlength{\parskip}{0.2\onelineskip} %\setlength{\parskip}{0.7em}

% Fin. Corrections automatiques de la mise en page
%\checkandfixthelayout[nearest] % Ajuste la hauteur du bloc de texte à un multiple entier de lignes
\checkandfixthelayout % application immédiate

% Style de page minimaliste personnalisé
\makepagestyle{minimaliste}
\makeevenhead{minimaliste}{}{}{\itshape\leftmark} % chapitre à droite (pages paires)
\makeoddhead{minimaliste}{\itshape\rightmark}{}{} % section à gauche (pages impaires)
\makeevenfoot{minimaliste}{\thepage}{}{} % numéro page à gauche en bas (pages paires)
\makeoddfoot{minimaliste}{}{}{\thepage} % numéro page à droite en bas (pages impaires)

% Pages paires : numéro en bas à gauche
\makeevenfoot{plain}{\thepage}{}{}
% Pages impaires : numéro en bas à droite
\makeoddfoot{plain}{}{}{\thepage}

% Style des titres (sobre) % maîtrise fine
%\usepackage{titlesec}
%\titleformat{\chapter}[hang]{\Large\bfseries\titlefont}{}{0pt}{}
%\titleformat{\section}[hang]{\large\bfseries}{}{0pt}{}
%\titleformat{\subsection}[runin]{\bfseries}{}{0pt}{}
%\titlespacing*{\chapter}{0pt}{20pt}{10pt}
%\titlespacing*{\section}{0pt}{12pt}{6pt}

% Style des titres natif à memoir (simple et stable)
\chapterstyle{veelo} % ou 'default', 'veelo', 'companion', 'ell', 'demo2', 'dash', etc.

\addto\captionsfrench{%
  \renewcommand{\listfigurename}{Liste des illustrations}%
  \renewcommand{\listtablename}{Liste des tableaux}%
}

% Mettre paragraph et subparagraph sur leur propre ligne
%\setbeforeparaskip{1.5ex}   % espace avant paragraph
%\setafterparaskip{1.0ex}    % espace après paragraph
%\setparaheadstyle{\normalfont\bfseries}    % paragraphe en gras
%\setbeforeparheadhook{}   
%\setafterparheadhook{\par}  % forcer saut de ligne

%\setbeforesubparaskip{1.2ex}
%\setaftersubparaskip{0.8ex}
%\setsubparaheadstyle{\normalfont\itshape}
%\setbeforesubparaheadhook{} 
%\setaftersubparaheadhook{\par}

% ---- Cadre encart remerciements ----
\tcbset{
  gratitude/.style={
    colframe=gray!50,        % couleur du contour
    colback=white,           % fond
    boxrule=0.3pt,           % épaisseur du trait
    arc=2pt,                 % coins légèrement arrondis
    left=8pt, right=8pt, top=6pt, bottom=6pt,
    width=0.75\textwidth,    % largeur fixe
    enhanced jigsaw,         % rendu propre
    before skip=1cm,         % espace au-dessus
    after skip=0pt,          % espace en-dessous
    overlay={%
      \node[anchor=north west, xshift=-0.6cm, yshift=0.6cm] at (frame.north east)
        {\includegraphics[width=0.9cm]{0_Liminaires/Images/EtoilePP.png}}; % étoile qui déborde
    }
  }
}