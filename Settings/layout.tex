\usepackage[pass]{geometry} % Pour des mises en forme local dans mon doc sans prise de tête liminaires et articles
% ---- Utilisation des fonctionnalités de la classe memoir ----
% 1. Taille du papier (définie lors de l'appel à la class memoir
%\settrims{0pt}{0pt} %\settrims{<haut>}{<bord_de_coupe>} en cas de bords perdus, en définir 2

% 2. Bloc de texte & marges
%\settypeblocksize{<hauteur>}{<largeur>}{<ratio>}
%\setlrmargins{⟨spine⟩}{⟨edge⟩}{⟨ratio⟩}
%\setulmargins{<haut>}{<bas>}{<ratio>} % Utilisez cette commande si la hauteur de votre bloc de texte est déjà fixe (par \settypeblocksize) et que vous positionnez le bloc en ajustant les marges, ex : \settypeblocksize{41\baselineskip + \topskip}{33pc}{*}
\setlrmarginsandblock{2.5cm}{5cm}{*} % Marges latérales (gauche/droite)
\setulmarginsandblock{3cm}{3cm}{*} % Marges haut/bas

% 3. En-têtes et pieds de page
\setheadfoot{\onelineskip}{2\onelineskip} % \setheadfoot{<hauteur_entete>}{<espacement_pied_de_page>}
%\setheaderspaces{1.5\onelineskip}{*} % \setheaderspaces{<chute_entete>}{<separation_entete_texte>}{<ratio>}
\setlength{\headheight}{15pt}

% 4. Notes marginales
\setmarginnotes{1em}{3.5cm}{1.5em} %\setmarginnotes{<separation>}{<largeur>}{<poussee>}

% Fin. Corrections automatiques de la mise en page
%\checkandfixthelayout[nearest] % Ajuste la hauteur du bloc de texte à un multiple entier de lignes
%\checkandfixthelayout % application immédiate

% Interligne plus aéré
\linespread{1.05}   % Memoir intègre en interne la gestion de l'interlignage...

% Paragraphe : pas d'indentation + saut entre paragraphes
%\setlength{\parindent}{0pt}
%\setlength{\parskip}{0.7em}
\setlength{\parskip}{0.2\onelineskip}

\checkandfixthelayout % application immédiate

% Améliorations typographiques
\usepackage{microtype} % césure fine, justification plus jolie

% Style de page minimaliste personnalisé
\makepagestyle{minimaliste}
\makeevenhead{minimaliste}{}{}{\itshape\leftmark} % chapitre à droite (pages paires)
\makeoddhead{minimaliste}{\itshape\rightmark}{}{} % section à gauche (pages impaires)
\makeevenfoot{minimaliste}{\thepage}{}{} % numéro page à gauche en bas (pages paires)
\makeoddfoot{minimaliste}{}{}{\thepage} % numéro page à droite en bas (pages impaires)

\makepagestyle{articlepage}
\makeevenhead{articlepage}{\small\itshape\leftmark}{}{}
\makeoddhead{articlepage}{}{}{\small\itshape\rightmark}
\makeevenfoot{articlepage}{\thepage}{}{}
\makeoddfoot{articlepage}{}{}{\thepage}

% Style des titres (sobre) % maîtrise fine
%\usepackage{titlesec}
%\titleformat{\chapter}[hang]{\Large\bfseries\titlefont}{}{0pt}{}
%\titleformat{\section}[hang]{\large\bfseries}{}{0pt}{}
%\titleformat{\subsection}[runin]{\bfseries}{}{0pt}{}
%\titlespacing*{\chapter}{0pt}{20pt}{10pt}
%\titlespacing*{\section}{0pt}{12pt}{6pt}

% Style des titres natif à memoir (simple et stable)
%\chapterstyle{section} % ou 'default', 'dash', 'hangnum', etc.

% Mettre paragraph et subparagraph sur leur propre ligne
%\setbeforeparaskip{1.5ex}   % espace avant paragraph
%\setafterparaskip{1.0ex}    % espace après paragraph
%\setparaheadstyle{\normalfont\bfseries}    % paragraphe en gras
%\setbeforeparheadhook{}   
%\setafterparheadhook{\par}  % forcer saut de ligne

%\setbeforesubparaskip{1.2ex}
%\setaftersubparaskip{0.8ex}
%\setsubparaheadstyle{\normalfont\itshape}
%\setbeforesubparaheadhook{} 
%\setaftersubparaheadhook{\par}

%\usepackage{geometry}
%\geometry{
%  top=2.5cm,
%  bottom=2.5cm,
%  inner=2.8cm,
%  outer=3.5cm,
%  bindingoffset=1cm,
%  includehead % Permet de conserver un espacement de 2.5 par rapport aux en-têtes plutôt que les bords de page.
%}