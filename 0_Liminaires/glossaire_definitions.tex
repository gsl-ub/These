% ---------- MOTS ---------- %
\newglossaryentry{accession fine}{
    name={Accession fine}, 
    text={accession fine}, 
    description={Accession ou lignée de vigne donnant des grappes millerandées, c'est-à-dire avec des baies de tailles variables et un port lâche. Plus le degré de finesse est élevé, plus l'hétérogénéité est grande},
    plural={accessions fines}
}

\newglossaryentry{baguette}{
    name={Baguette}, 
    text={baguette}, 
    description={Terme viticole, notamment dans la taille en Guyot, désignant le bois de l'année précédente laissé sur le pied pour porter les rameaux fructifères de l'année suivante},
    plural={baguettes}
}

% ---------- ACRONYMES ---------- %
\newacronym{aba}{ABA}{Acide abscissique}
\newacronym{acp}{ACP}{Analyse en composantes principales}

% ---------- MODE D'EMPLOI ---------- %
% Une fois un mot ou acronyme déclaré avec \newglossaryentry ou \newacronym :
% Pour l’utiliser dans ton texte :
%\gls{nmr} → affiche NMR + explication complète la 1re fois (si configuré)

%\Gls{nmr} → idem, mais avec majuscule

%\glspl{nmr} → pluriel

%\acrshort{nmr} → affiche juste NMR

%\acrfull{nmr} → affiche acide abscissique (ABA) (forme longue + sigle)

%\acrlong{nmr} → juste la forme longue

% LaTeX se charge automatiquement d’écrire le mot complet la première fois, puis seulement l’abréviation ensuite.}