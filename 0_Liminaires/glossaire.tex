% Mise en forme du glossaire contenu dans glossaire_definitions.tex

%\chapter*{Glossaire} % crée un chapitre non numéroté, il n’est pas ajouté automatiquement à la table des matières
%\addcontentsline{toc}{chapter}{Glossaire} % Cette commande ajoute manuellement une ligne dans la table des matières (ToC). toc = table of contents ; chapter = niveau hiérarchique de l’entrée (ici, un chapitre) ; Glossaire = le texte à afficher dans la ToC

% Glossaire et acronymes — à inclure avec % Mise en forme du glossaire contenu dans glossaire_definitions.tex

%\chapter*{Glossaire} % crée un chapitre non numéroté, il n’est pas ajouté automatiquement à la table des matières
%\addcontentsline{toc}{chapter}{Glossaire} % Cette commande ajoute manuellement une ligne dans la table des matières (ToC). toc = table of contents ; chapter = niveau hiérarchique de l’entrée (ici, un chapitre) ; Glossaire = le texte à afficher dans la ToC

% Glossaire et acronymes — à inclure avec % Mise en forme du glossaire contenu dans glossaire_definitions.tex

%\chapter*{Glossaire} % crée un chapitre non numéroté, il n’est pas ajouté automatiquement à la table des matières
%\addcontentsline{toc}{chapter}{Glossaire} % Cette commande ajoute manuellement une ligne dans la table des matières (ToC). toc = table of contents ; chapter = niveau hiérarchique de l’entrée (ici, un chapitre) ; Glossaire = le texte à afficher dans la ToC

% Glossaire et acronymes — à inclure avec % Mise en forme du glossaire contenu dans glossaire_definitions.tex

%\chapter*{Glossaire} % crée un chapitre non numéroté, il n’est pas ajouté automatiquement à la table des matières
%\addcontentsline{toc}{chapter}{Glossaire} % Cette commande ajoute manuellement une ligne dans la table des matières (ToC). toc = table of contents ; chapter = niveau hiérarchique de l’entrée (ici, un chapitre) ; Glossaire = le texte à afficher dans la ToC

% Glossaire et acronymes — à inclure avec \include{Liminaires/glossaire}

% Affiche la liste des acronymes
\printglossary[
  type=\acronymtype,
  title=Acronymes,
  toctitle=Acronymes % Alternative : Liste des acronymes, changera le titre uniquement dans la toc (table of content)
]

% Affiche le glossaire principal
\printglossary[
  title=Glossaire,
  toctitle=Glossaire % Glossaire des termes
]

% Utilisation à supprimer
Une fois un mot ou acronyme déclaré avec \newglossaryentry ou \newacronym :
Pour l’utiliser dans ton texte :
\gls{nmr} → affiche NMR + explication complète la 1re fois (si configuré)

\Gls{nmr} → idem, mais avec majuscule

\glspl{nmr} → pluriel

\acrshort{nmr} → affiche juste NMR

\acrfull{nmr} → affiche acide abscissique (ABA) (forme longue + sigle)

\acrlong{nmr} → juste la forme longue

LaTeX se charge automatiquement d’écrire le mot complet la première fois, puis seulement l’abréviation ensuite.

% Affiche la liste des acronymes
\printglossary[
  type=\acronymtype,
  title=Acronymes,
  toctitle=Acronymes % Alternative : Liste des acronymes, changera le titre uniquement dans la toc (table of content)
]

% Affiche le glossaire principal
\printglossary[
  title=Glossaire,
  toctitle=Glossaire % Glossaire des termes
]

% Utilisation à supprimer
Une fois un mot ou acronyme déclaré avec \newglossaryentry ou \newacronym :
Pour l’utiliser dans ton texte :
\gls{nmr} → affiche NMR + explication complète la 1re fois (si configuré)

\Gls{nmr} → idem, mais avec majuscule

\glspl{nmr} → pluriel

\acrshort{nmr} → affiche juste NMR

\acrfull{nmr} → affiche acide abscissique (ABA) (forme longue + sigle)

\acrlong{nmr} → juste la forme longue

LaTeX se charge automatiquement d’écrire le mot complet la première fois, puis seulement l’abréviation ensuite.

% Affiche la liste des acronymes
\printglossary[
  type=\acronymtype,
  title=Acronymes,
  toctitle=Acronymes % Alternative : Liste des acronymes, changera le titre uniquement dans la toc (table of content)
]

% Affiche le glossaire principal
\printglossary[
  title=Glossaire,
  toctitle=Glossaire % Glossaire des termes
]

% Utilisation à supprimer
Une fois un mot ou acronyme déclaré avec \newglossaryentry ou \newacronym :
Pour l’utiliser dans ton texte :
\gls{nmr} → affiche NMR + explication complète la 1re fois (si configuré)

\Gls{nmr} → idem, mais avec majuscule

\glspl{nmr} → pluriel

\acrshort{nmr} → affiche juste NMR

\acrfull{nmr} → affiche acide abscissique (ABA) (forme longue + sigle)

\acrlong{nmr} → juste la forme longue

LaTeX se charge automatiquement d’écrire le mot complet la première fois, puis seulement l’abréviation ensuite.

% Affiche la liste des acronymes
%\printacronyms[title=Acronymes,]

% Affiche le glossaire principal
%\printglossary[title=Glossaire, type=main]

% Utilisation [à supprimer]
%Une fois un mot ou acronyme déclaré avec \newglossaryentry ou \newacronym :
%Pour l’utiliser dans ton texte :
%\gls{nmr} → affiche NMR + explication complète la 1re fois (si configuré)

%\Gls{nmr} → idem, mais avec majuscule

%\glspl{nmr} → pluriel

%\acrshort{nmr} → affiche juste NMR

%\acrfull{nmr} → affiche acide abscissique (ABA) (forme longue + sigle)

%\acrlong{nmr} → juste la forme longue

%LaTeX se charge automatiquement d’écrire le mot complet la première fois, puis seulement l’abréviation ensuite.