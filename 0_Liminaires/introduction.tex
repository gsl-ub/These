\addcontentsline{toc}{chapter}{Introduction générale}
\chapter*{Introduction générale}

...

Des analyses de répétabilité sont réalisées sur le spectromètre \textbf{500 MHz} avec des mesures \textbf{inter-day} en conditions de fidélité intermédiaire.
\textbf{Attention} à la comparaison 400 MHz / 500 MHz : certaines corrections pourraient manquer sur le \textbf{400 MHz}. Ne pas utiliser des données 400 et faire la démonstration uniquement sur la base du 500 MHz.

\subsubsection{Détermination profils d'exactitude des composés : incertitudes de mesure et LOD/LOQ}
Plusieurs méthodes d'estimation des limites de détection (LOD) et limites de quantification (LOQ) sont comparées, appliquées à un ou plusieurs composés -> Thèse

Présenter les profils d'exactitude desquels on peut déduire une incertitude relative ainsi que les LOQ (à 60 \% d'incertitude) et LOD. Proposer une estimation sur la base du S/N et des essais sur la répétabilité lorsque ce n'est pas possible.