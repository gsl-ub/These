\addcontentsline{toc}{section}{Citations}
%\section*{Citations}

\vspace*{3 cm}

\begin{center}

\begin{quote} \textit{\og All models are wrong, but some are useful. \fg{}}\\
\textit{(Tous les modèles sont faux, mais certains sont utiles.)}%
\vspace{0.8\baselineskip} % 0.8 fois l'interligne
\sourceatright{— George E. P. Box, \textit{Empirical Model-Building and Response Surfaces}} \end{quote}%

\vspace{1.5 cm}

%\begin{quote} \textit{\og La théorie, c'est quand on sait tout et que rien ne fonctionne. La pratique, c'est quand tout fonctionne et que personne ne sait pourquoi. Ici, nous avons réuni théorie et pratique : Rien ne fonctionne... et personne ne sait pourquoi ! \fg{}}\\
%\vspace{0.8\baselineskip} % 0.8 fois l'interligne
%\sourceatright{— \textit{souvent attribué à} Albert Einstein} \end{quote}

%\vspace{1.5 cm}

%\begin{quote} \textit{\og La créativité, c'est l'intelligence qui s'amuse. \fg{}}\\
%\vspace{0.8\baselineskip} % 0.8 fois l'interligne
%\sourceatright{— Albert Einstein} \end{quote}

%\begin{quote} \textit{\og La logique vous mènera de A vers B. L'imagination vous emmènera partout. \fg{}}\\
%\vspace{0.8\baselineskip} % 0.8 fois l'interligne
%\sourceatright{— Albert Einstein} \end{quote}

%\vspace{1.5 cm}

%\vspace{1.5 cm}

%\begin{quote} \textit{\og La simplicité est la sophistication suprême. \fg{}}\\
%\vspace{0.8\baselineskip} % 0.8 fois l'interligne
%\sourceatright{— Léonard de Vinci} \end{quote}

%\vspace{1.5 cm}

%\begin{quote} \textit{\og La perfection est atteinte, non pas lorsqu'il n'y a plus rien à ajouter, mais lorsqu'il n'y a plus rien à retirer. %\fg{}}\\
%\vspace{0.8\baselineskip} % 0.8 fois l'interligne
%\sourceatright{— Antoine de Saint-Exupéry} \end{quote}

%\vspace{1.5 cm}

%\begin{quote} \textit{\og Tout doit être rendu aussi simple que possible, mais pas plus. \fg{}}\\
%\vspace{0.8\baselineskip} % 0.8 fois l'interligne
%\sourceatright{— Albert Einstein} \end{quote}

%\vspace{1.5 cm}

%\begin{quote} \textit{\og Il faut exiger de chacun ce que chacun peut donner. \fg{}}\\
%\vspace{0.8\baselineskip} % 0.8 fois l'interligne
%\sourceatright{— Antoine de Saint-Exupéry} \end{quote}

%\vspace{1.5 cm}

\begin{quote} \textit{\og On ne voit bien qu'avec le c\oe ur. L'essentiel est invisible pour les yeux. \fg{}}%
\vspace{0.8\baselineskip} % 0.8 fois l'interligne
\sourceatright{— Antoine de Saint-Exupéry, \textit{Le Petit Prince}} \end{quote}

%\vspace{1.5 cm}

%\begin{quote} \textit{\og Fais de ta vie un rêve, et d'un rêve, une réalité. \fg{}}\\
%\vspace{0.8\baselineskip} % 0.8 fois l'interligne
%\sourceatright{— Antoine de Saint-Exupéry} \end{quote}

\vfill % pousse ce qui suit en bas de page
\includegraphics[width=0.25\textwidth]{0_Liminaires/Images/2025-08-10_ChatGPT.png} % image centrée, largeur ajustable % pour aligner à droite, sortir du center et faire un \hfill

\end{center}

\newpage
%\cleardoublepage
\addcontentsline{toc}{section}{Remerciements}

\section*{Remerciements}

%\vspace*{1.3\baselineskip}
% ----- encart définition -----
\begin{quote}\textit{\og \textbf{Gratitude} (n.f.) : Sentiment de reconnaissance envers une personne dont on est l'obligé, à qui l'on sait gré d'un bienfait, d'un service rendu. \fg{}}%
\vspace{0.5\baselineskip} % 0.8 fois l'interligne
\sourceatright{--- Dictionnaire de l'Académie française, \textit{9\textsuperscript{e} édition}} \end{quote}%
%\begin{quote} \textit{\og \textbf{Gratitude} (n.f.) : Reconnaissance pour un service, pour un bienfait reçu ; sentiment affectueux envers un bienfaiteur : Manifester sa gratitude à quelqu'un. \fg{}}\\
%\vspace{0.8\baselineskip} % 0.8 fois l'interligne
%\sourceatright{--- Larousse} \end{quote}

\begin{center}\begin{tcolorbox}[gratitude]
%\noindent\makebox[\textwidth][c]{\begin{minipage}{0.75\textwidth}\raggedright
    L'esprit libre et le \textit{c\oe ur} ouvert, %\\[0.15\baselineskip] % alternative : et avec mon \textit{c\oe ur},
    je fais don de ma \textit{gratitude}\\[0.15\baselineskip] % ajout : dans sa forme la plus pure
    à celles et ceux qui, dans cette étape de ma vie, \\[0.15\baselineskip]
    ont laissé dans mon ciel des étoiles éternelles.
%    \end{minipage}}%
\end{tcolorbox}\end{center}%

{\setlength{\parskip}{0.4em}\setlength{\parindent}{0.8em}\begingroup\setSingleSpace{1.06}\SingleSpacing%
Tout d'abord, j'adresse ma reconnaissance aux membres de mon jury\,--\,mes premiers lecteurs\,--\,\dots\ mes rapporteurs et \dots\ mes examinateurs. Merci d'avoir accepté de consacrer du temps et de l'intérêt à l'étude de mes travaux et de contribuer à ma quête constante d'amélioration.

La suite naturelle de ces remerciements va à mon directeur de thèse, le professeur Tristan Richard ainsi qu'à mes deux co-encadrants, les docteurs Grégory Da Gosta et Josep Valls-Fonayet.\\[0.15\baselineskip]Tristan, votre confiance, votre bienveillance et votre soutien tout au long de ce parcours scientifique m'ont touché bien au-delà de ce que je ne saurais exprimer avec des mots. En hommage à l'un de vos conseils les plus précieux, j'irai à l'essentiel~: \textit{merci}.\\
Greg, guide du quotidien, merci d'avoir accompagné mes premiers pas en RMN et d'avoir pourvu à ma curiosité insatiable, répondant aussi bien à mes questions les plus pointues qu'aux plus improbables avec patience --- comme le jour où j'ai oublié d'ajouter le solvant deutéré dans mon échantillon\dots\ (car oui, il m'arrive d'être aussi étourdi.)\\Josep, merci de m'avoir ouvert les portes des arcanes de l'analyse LC-MS et d'y avoir accueilli un jeune RMNiste avec chaleur et générosité.

Après ces mots personnels, je tiens à remercier les femmes et les hommes qui animent les institutions ayant rendu ces travaux possibles~: l'Université de Bordeaux, l'école doctorale Sciences de la Vie et de la Santé, et mon laboratoire d'accueil, l'UMR \OE no. Merci à mes tuteurs Cédric Saucier et Sophie Colombié pour leur suivi, ainsi qu'à l'équipe pédagogique de l'ISVV pour m'avoir initié à l'enseignement universitaire deux années de suite. Merci à Patrick Lucas, directeur de l'UMR, pour ses encouragements lorsque je suis revenue à mon amour premier~: \textit{la science}.

Je remercie les membres du projet WAP NMR, au sein duquel ma thèse a pris vie. Merci à Matthieu Dubernet pour vos conseils avisés et la réactivité de vos équipes, à Alexandra Gosse des Grands Chais de France pour les échantillons fournis, à Jean-Claude Boulet pour ton investissement, ainsi qu'à Laeticia Gaillard, Antoine Galvan, Sophie Rosset et Jean-Phillipe Rosec du SCL de Pessac pour nos échanges. Un merci chaleureux à Catherine Deborde, qui m'a accompagnée dans mes seconds pas en RMN avec le 500~MHz et son passeur automatique -- un apprivoisement qui fut à la fois une bataille et une révolution dans mon quotidien, suite au quench de notre regretté 600~MHz. Enfin, Daniel Jacob, je t'adresse mes remerciements sincères pour ta disponibilité et ta contribution clé à l'automatisation du traitement des spectres RMN, élément crucial pour la réalisation de ces travaux.

Thank you to the the Australian Wine Research Institute members for your warm welcome and the collaboration we have shared. Markus Herderich, Flynn Watson, Luca Nicolotti and Natoya Loyd -- working with you has been, is, and will remain, a joy and a privilege.

Je souhaite également remercier les membres du Réseau Francophone de Métabolomique et de Fluxomique pour la richesse de nos échanges. Je pense tout particulièrement à Fabien Jourdan qui a su me rassurer dans les périodes de doute, à travers le programme de mentorat.

À ma \og famille scientifique \fg{}, les membres de l'axe Molécule d'Intérêt Biologique (MIB pour les intimes), merci de m'avoir accueilli avec tant de bienveillance et de faire de notre équipe un espace où il fait bon travailler et partager. Karen, Stéphanie K., Stéphanie C., Arnaud, Eva, Mathilde, Caroline, David, Anthony, Lisa, Wiktoria, Marie-Laure, Antonio, Marina, merci ces discussions et ces éclats de rire qui ont rythmés notre quotidien. J'ai eu la chance d'accueillir Rémi, d'abord stagiaire puis ingénieur d'étude : ton implication et tes qualités humaines ont été des atouts précieux, autant dans le travail qu'en amitié. Plus récemment, Lou-Ann nous a rejoint pour son stage de master 2, apportant sa contribution avec sérieux et enthousiasme. Je tiens également à remercier Andrew, Patricia (It may not be Thursday, yet I will indulge in French for these acknowledgements), Luce et Sara pour ces instants partagés~: brunchs, cafés, changements de langue à la volée, confidences comme fous rires~; autant de moments qui ont forgés ces amitiées a jamais gravées dans ma mémoire.

Au sortir du laboratoire, je n'oublie pas ceux qui ont su apprivoiser nos décalages -- surtout les miens. Ceux-là même pour qui un sobre \textit{merci} suffirait à trouver sa place dans nos c\oe urs, mais qui me savent bien trop volubile pour m'en contenter. Antoine, ton humour délicieusement douteux et ma personnalité joyeusement singulière font bon ménage depuis longtemps~; gardons ensemble nos âmes d'enfants dans un monde toujours plus adulte. Théo, notre histoire, dont les graines ont été semées à l'école, reste toujours aussi captivante. Florian, merci de ton accueil~; je garderai à jamais le souvenir de soirées incroyables... et de la leçon que le casino n'est pas pour moi -- je suis déjà bien assez chanceux par ailleurs dans ma vie. Charles, nos rares retrouvailles semblent toujours dater d'hier. Merci de me garder une place dans ton ciel d'étoiles et pour avoir illustré ces premières pages d'un vibrant hommage au \textit{Petit Prince} qui nous émeut tant. Enfin, Yoann, chaque moment partagé depuis notre rencontre sont autant de parenthèses hors du temps dont je chéris l'existence. Dans chacun de nos échanges, je trouve de l'inspiration pour mieux grandir.

À Paul, ton amour et ta bonté semblent échapper à toute mesure, chaque jour, l'infini prend une nouvelle saveur à tes côtés. Quoi qu'il arrive, merci pour chacun de ces instants passés, présents et à venir. À celle qui ma mis au monde avec amour~: ma maman, je t'aime. Mamie, merci d'avoir veillé au grain à la sortie de l'école, en insistant lorsqu'il fallait apprivoiser l'âpre mais si beau français. Je garde en moi la trace indélébile de la bienveillance de feu mon papi, source d'inspiration. Papa, merci d'avoir toujours su m'entourer de ton amour. Emma, ton âme sait transmettre des émotions si pures qu'elles m'émerveilleront toujours. Marraine, ta force de vie et ta sagesse --~pas toujours si sage~-- illuminent mon chemin depuis notre rencontre~; je suis sûr que tu sauras ce que \textit{merci} signifie entre nous.

À ceux qui ne se retrouveraient pas dans ces pages, sachez que cela n'amoindrit en rien la place que vous occupez dans mon c\oe ur. Lui, il n'oublie jamais les âmes avec qui je suis entré en résonnance. Ma mémoire, en revanche, se montre parfois capricieuse -- surtout en ces intenses moments de rédaction.

Une fois encore, je risque de finir affublé du qualificatif \og loquace \fg{} (à raison, je le concède). Cependant il est un dernier remerciement que je ne puis oublier~; \textit{cher lecteur}, veuillez recevoir ma profonde gratitude. C'est dans le don discret de votre attention que naît le sens ces pages.
\endgroup}%

% Maintenant déballe ton CV
\newpage
\addcontentsline{toc}{section}{Productions scientifiques}
\section*{Productions scientifiques}
\subsection*{Communications orales}
\begin{itemize}
    \item Workshop Modélisation 2024 - Toulouse (17/03/2024)
    \item ...
\end{itemize}

\subsection*{Posters scientifiques}

\begin{itemize}
    \item Journées Jeunes Chercheurs 2023 - Clermont-Ferrand (03/04/2023)
    \item ...
\end{itemize}

\vspace{10pt}

\subsection*{Activités annexes à la recherche}
\begin{itemize}
    \item Encadrement de stagiaires
    \item ADPI...
\end{itemize}