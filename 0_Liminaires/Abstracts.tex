%{\newgeometry{top=2.5cm,bottom=2.5cm,left=2.5cm,right=2.5cm,includehead=false}
% --- Résumé FR ---
%\thispagestyle{empty}
\addcontentsline{toc}{section}{Résumé}
\begin{small}
{
\linespread{1.05}\selectfont
\begin{center}
\textbf{La métabolomique au service de l’authenticité des vins : analyses RMN-$^1$H, quantifications ciblées et fusion de données multiplateformes}
\end{center}

\noindent\textbf{Résumé :} % maximum 1700 caractères, espaces compris recommandation nationale
Le secteur vitivinicole européen génère une production de richesse importante, mais il est confronté à un problème croissant de fraudes et de contrefaçons. La falsification des vins constitue l’une des fraudes alimentaires les plus anciennes et les plus répandues à l’échelle mondiale. Dès lors, la question de l’authenticité des vins, étroitement liée à celle de la traçabilité, demeure un enjeu majeur.

Pour y répondre, des méthodes analytiques robustes et fiables sont nécessaires. Celles-ci vont de techniques simples, telles que les analyses organoleptiques, à des approches plus avancées, comme les mesures isotopiques pour la détermination de l’origine botanique et géographique ou la détection de la radioactivité à très bas bruit pour la datation.

Toutefois, face à la recrudescence et à la sophistication des fraudes, il devient nécessaire de développer des stratégies analytiques plus élaborées. Dans ce contexte, les approches métabolomiques, en particulier par spectroscopie RMN (Résonance Magnétique Nucléaire) et spectrométrie de masse (MS), offrent des perspectives prometteuses pour renforcer le contrôle de l’authenticité des vins.

L’objectif principal de ces travaux consiste à développer des outils de contrôle fondés sur des approches métabolomiques RMN-$^1$H, à valider la quantification de composés dans les vins par ces méthodes et à améliorer le suivi de la traçabilité en intégrant des analyses complémentaires telles que la LC-MS par des stratégies de fusion des données. Ces travaux sont menés dans une optique d’application au contrôle officiel des vins.

\vspace{0.8\baselineskip} % ou 0.5cm si tu veux une valeur fixe
\noindent\textbf{Mots-clés :} ...

\vspace{0.6\baselineskip}
\noindent\makebox[\linewidth]{\rule{\textwidth}{0.4pt}}
\vspace{0.6\baselineskip}

%\vspace{0.4cm}
%\noindent\makebox[\linewidth]{\rule{\textwidth}{0.4pt}}

%\vfill
%\begin{center}
%\textbf{Unité de recherche} \\
%UMR 1366 OENO, 33140 Villenave-d'Ornon, France.
%\end{center}
%\end{small}

%\cleardoublepage                     % forcer page de droite suivante

% --- Abstract EN ---
%\thispagestyle{empty}
%\addcontentsline{toc}{section}{Abstract}
%\begin{small}
%\linespread{1.05}\selectfont
\begin{center}
\textbf{Wine authenticity: metabolomic analyses by $^1$H-NMR, targeted quantification and multi-platform data fusion}
\end{center}

\noindent\textbf{Abstract:} 
The European wine sector represents a major source of economic value but is increasingly affected by fraud. Wine counterfeiting is one of the oldest and most widespread forms of food fraud. Therefore, the issue of wine authenticity, closely related to traceability, remains a critical concern.

Addressing this challenge requires reliable and robust analytical methods. These range from simple techniques, such as organoleptic assessments, to more advanced approaches, including isotopic measurements for determining botanical and geographical origin, or ultra-low-background radioactivity detection for wine dating.

However, with increasingly sophisticated falsification methods, the development of more complex analytical strategies has become essential. In this context, metabolomic approaches—particularly those based on Nuclear Magnetic Resonance (NMR) spectroscopy and Mass Spectrometry (MS)—offer promising perspectives to strengthen the control of wine authenticity.

The main objective of this work is to develop control tools based on $^1$H-NMR metabolomic approaches, to validate the quantification of compounds in wines using these methods, and to improve traceability monitoring by integrating complementary analyses—such as LC-MS—through data fusion strategies. This research is conducted with the aim of supporting official wine control.

\vspace{0.8\baselineskip} % ou 0.5cm si tu veux une valeur fixe
\noindent\textbf{Keywords:} FDsqlfvk, fsdqfml k,  fdsqml kf %\\

\vspace{0.4cm}
\noindent\makebox[\linewidth]{\rule{\textwidth}{0.4pt}}

\vfill
\begin{center}
\textbf{Unité de recherche --- Research unit} \\
UMR 1366 OENO -- Université de Bordeaux, INRAE -- 33140 Villenave-d'Ornon, France.
\end{center}
}
\end{small}

%\restoregeometry}
\cleardoublepage                     % forcer page de droite suivante