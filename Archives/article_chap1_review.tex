Faire ici une biblio dédiée, utiliser multicols 2, eventuellement utiliser refsection et une bibliographie associée.
Pour éviter soucis page gauche / droite peut être rebasculer sur des pages
Voir aussi possibilité du package subfiles ou inclure les PDF directement
% ----- Insertion locale d'un article -----

\clearpage % force une nouvelle page propre

% modification temporaire des marges locales
\newgeometry{margin=2cm}

% création d'une entête personnalisée simple via memoir
\renewcommand{\articletitle}{Cork impact on red wine aging}
\renewcommand{\articlejournal}{Food Res Int, 203 (2025)}

% application du style personnalisé local
\pagestyle{article}

% passage temporaire en double colonne
\twocolumn[
  \begin{@twocolumnfalse}
    % Contenu sur une seule colonne (Titre, auteurs, DOI, etc.)
    \begin{center}
        \section{Titre de l'article}
        %\Large\textbf{Titre de l'article}\\[1em]
        Auteurs\\[0.5em]
        \normalsize DOI: xxxx, Journal: xxx (année)
        \vspace{1em}
    \end{center}
    % Résumé
    \begin{abstract}
      Ton résumé ici.
    \end{abstract}
    \vspace{1em} % un peu d'espace
    \noindent \textbf{Mots-clés :} mot1, mot2, mot3, mot4.
    \vspace{2em} % un peu d'espace avant le contenu double colonne
  \end{@twocolumnfalse}
]

% contenu principal de l'article en double colonne
%\input{articles/article1.tex}
% Corps de l'article ici, en 2 colonnes automatiquement.
% Figures incluses normalement avec \begin{figure}...\end{figure}
\begin{figure}[htbp]
\centering
\includegraphics[width=\columnwidth]{1_Chapitre_1_Etat_Art/Figures/mafigure.pdf/png/jpeg/...}
\caption{Légende ici}
\end{figure}

% Bibliographie spécifique à l'article avec biblatex localement
{
\footnotesize % éventuellement réduire un peu la taille de la biblio locale sinon \small
\printbibliography[heading=subbibliography,keyword=article1] % nécessité de taguer les articles avec un keyword correspondant dans Zotero
}

% Fin de la mise en page locale
\onecolumn % retour à une colonne
\restoregeometry % restauration marges initiales
\pagestyle{minimaliste} % retour à ton style initial (que tu as défini)
\clearpage % force une page propre après l'article