% ============================
% Classe du document
% ============================
\documentclass[ % 'report' est bien adapté aux thèses mais peut être préférer 'memoir' ou equivalents 'scrbook', 'scrreprt' (Koma-script)
a4paper,
11pt,
twoside
]{memoir}

% ============================
% Chargement des réglages
% ============================
%% --- Langue et encodage (gérés par fontspec si LuaLaTeX/XeLaTeX) ---
\usepackage[english, french]{babel}         % Gestion multilingue, langue française par défaut (dernière) : typographie, noms automatiques, utiliser \selectlanguage{english} pour basculer en anglais et \begin{otherlanguage}{french} pour des inclusions fr suivi de \end{otherlanguage} puis de \selectlanguage{english} pour reprendre une partie en anglais.
\usepackage{csquotes}                       % Guillemets typographiques adaptés à la langue
\addto\captionsfrench{%
  \renewcommand{\listfigurename}{Liste des illustrations}%
  \renewcommand{\listtablename}{Liste des tableaux}%
}
\usepackage{multicol}                       % pour un environnement 2-colonnes
% A supprimer lorsque j'utiliserai des polices personnalisées.
\usepackage[T1]{fontenc}                    % Sortie correcte des accents
\usepackage{lmodern}                        % Police vectorielle moderne

% --- Mathématiques et symboles scientifiques ---
\usepackage{amsmath, amssymb, amsfonts}     % Pour les formules et symboles mathématiques
\usepackage{siunitx}                        % Gestion des unités scientifiques (1.23 en et 1,23 fr) \SI{3.5}{mg/L} -> possibilité aussi de faire des unités personnalisées voir chatGPT au besoin

% --- Images et figures ---
\usepackage{graphicx}                       % Insertion d'images
\usepackage{float}                          % Placement forcé avec [H]
\usepackage[font=small,labelfont=bf,width=0.8\textwidth]{caption}   % Style des légendes
\usepackage{subcaption}                     % Sous-figures
% \usepackage{svg}                          % Pour inclure des fichiers SVG

% --- Tableaux et couleurs ---
\usepackage[table,xcdraw]{xcolor}           % Couleurs dans les tableaux et texte
\usepackage{multirow}                       % Fusion de cellules
% \usepackage{booktabs}                     % Tableaux élégants si besoin (\toprule...)

% --- Bibliographie ---
\usepackage[backend=biber, style=apa, sorting=nyt, url=false]{biblatex} % style=iso-alphabetic, cas complexes ! norme AFNOR Z 44005 (reco nationale)
\addbibresource{references.bib}             % Fichier de bibliographie
% \addbibresource{biblio_manuelle.bib}      % Autre fichier possible -> utiliser betterbiblatex... ?
% Réglages typographiques utiles
\setcounter{biburllcpenalty}{7000}
\setcounter{biburlucpenalty}{8000}
%Augmente le seuil de coupure dans les URL en bibliographie. Cela évite les césures moches dans les longues URL.
%\renewcommand*{\labelnamepunct}{\newunitpunct\par}% Met un saut de ligne après les noms d’auteur au lieu d’un point, pour une bibliographie plus aérée.
%\renewbibmacro{in:}{\newline In:}           % Met également un saut de ligne avant le mot In: (chapitre dans un ouvrage), pour un rendu plus clair.
%\DefineBibliographyExtras{french}{\restorecommand\mkbibnamefamily} % Restaure un comportement cohérent en français pour la bibliographie dans la mise en forme des noms de famille

% --- Glossaire et acronymes ---
\usepackage[toc, acronym]{glossaries}
% ---------- MOTS ---------- %
\newglossaryentry{accession fine}{
    name={Accession fine}, 
    text={accession fine}, 
    description={Accession ou lignée de vigne donnant des grappes millerandées, c'est-à-dire avec des baies de tailles variables et un port lâche. Plus le degré de finesse est élevé, plus l'hétérogénéité est grande},
    plural={accessions fines}
}

\newglossaryentry{baguette}{
    name={Baguette}, 
    text={baguette}, 
    description={Terme viticole, notamment dans la taille en Guyot, désignant le bois de l'année précédente laissé sur le pied pour porter les rameaux fructifères de l'année suivante},
    plural={baguettes}
}

% ---------- ACRONYMES ---------- %
\newacronym{aba}{ABA}{Acide abscissique}
\newacronym{acp}{ACP}{Analyse en composantes principales}

% ---------- MODE D'EMPLOI ---------- %
% Une fois un mot ou acronyme déclaré avec \newglossaryentry ou \newacronym :
% Pour l’utiliser dans ton texte :
%\gls{nmr} → affiche NMR + explication complète la 1re fois (si configuré)

%\Gls{nmr} → idem, mais avec majuscule

%\glspl{nmr} → pluriel

%\acrshort{nmr} → affiche juste NMR

%\acrfull{nmr} → affiche acide abscissique (ABA) (forme longue + sigle)

%\acrlong{nmr} → juste la forme longue

% LaTeX se charge automatiquement d’écrire le mot complet la première fois, puis seulement l’abréviation ensuite.}
\makeglossaries
% \renewcommand*{\glsentryfmt}[1]{#1}       % Désactiver l'insertion automatique de la définition dans le texte

% --- Commentaires en marge (optionnels) ---
\usepackage[textsize=footnotesize, color=orange!20]{todonotes}

% --- Gestion des annexes ---
%\usepackage{appendix}                       % Gestion propre des annexes : inutile avec la classe memoir

% --- Hyperliens ---
\usepackage[colorlinks=true,
            linkcolor=black,
            citecolor=black,
            filecolor=black,
            urlcolor=blue]{hyperref}
%\usepackage[hidelinks]{hyperref}            
% \usepackage{cleveref}                     % Références intelligentes (ex: \cref{fig:truc})               % chargement des packages utilisés
%% Utilisation de polices système/personnalisées avec fontspec
% Pour utiliser fontspec (nécessite xelatex ou lualatex) -> Xualatex mieux
\usepackage{fontspec}

% Chemin vers les polices
\defaultfontfeatures{Path=Fonts/}

% Chemin vers les polices personnalisées (placer les .otf dans Fonts/ à la racine)
\defaultfontfeatures{Path=Fonts/}

% Police principale du corps du texte
\setmainfont{Produkt-Regular.otf}[
  Extension = .otf,
  UprightFont = * ,
  BoldFont = Produkt-Bold,
  ItalicFont = Produkt-Italic,
  BoldItalicFont = Produkt-BoldItalic
]
\setmainfont{EB Garamond}[    % nom exact du fichier ou de la police installée
  UprightFont      = *-Regular,
  ItalicFont       = *-Italic,
  BoldFont         = *-Bold,
  BoldItalicFont   = *-BoldItalic,
  Extension        = .otf
]

% Police pour les titres ou parties spécifiques
\newfontfamily\titlefont{AvenirNext-Regular.otf}[
  Extension = .otf,
  UprightFont = * ,
  BoldFont = AvenirNext-Bold
]                  % utilisation de polices personnalisées
%% Marges asymétriques pour impression recto-verso
\usepackage{geometry}
\geometry{
  top=2.5cm,
  bottom=2.5cm,
  inner=2.8cm,
  outer=3.5cm,
  bindingoffset=1cm,
  includehead % Permet de conserver un espacement de 2.5 par rapport aux en-têtes plutôt que les bords de page.
}

% Interligne plus aéré
%\usepackage{setspace}   % Utilisation de \setstretch{1.05} pour la gestion des interlignes.
%\setstretch{1.05} % Reglage des interlignes
\linespread{1.05}   % Memoir intègre en interne la gestion de l'interlignage...

% Paragraphe : pas d'indentation + saut entre paragraphes
\setlength{\parindent}{0pt}
\setlength{\parskip}{0.7em}

% Améliorations typographiques
\usepackage{microtype} % césure fine, justification plus jolie

% Style des titres (sobre) % maîtrise fine
%\usepackage{titlesec}
%\titleformat{\chapter}[hang]{\Large\bfseries\titlefont}{}{0pt}{}
%\titleformat{\section}[hang]{\large\bfseries}{}{0pt}{}
%\titleformat{\subsection}[runin]{\bfseries}{}{0pt}{}
%\titlespacing*{\chapter}{0pt}{20pt}{10pt}
%\titlespacing*{\section}{0pt}{12pt}{6pt}

% Style des titres natif à memoir (simple et stable)
%\chapterstyle{section} % ou 'default', 'dash', 'hangnum', etc.

% Appliquer la bonne police (définie comme \titlefont dans fonts.tex) aux titres de chapitres
%\renewcommand{\chapnamefont}{\titlefont\LARGE}  % "Chapitre"
%\renewcommand{\chapnumfont}{\titlefont\LARGE}   % Numéro
%\renewcommand{\chaptitlefont}{\titlefont\LARGE} % Titre
% tailles polices : \tiny < \scriptsize < \footnotesize < \small < \normalsize < \large < \Large < \LARGE < \huge < \Huge
% graisse : \bfseries

% Numérotation continue des sections (optionnel)
% \counterwithout{section}{chapter} % décommente si tu ne veux pas 1.1, 1.2, mais juste 1, 2, 3

% En-têtes et pieds de page classiques (peu chargés)
\pagestyle{plain} % pas d'en-tête et numerotation en bas centrée.
% Possibilité avec le package fancyhdr de maîtriser les entêtes... plus tard.

% ============================
% Mise en page - OLD
% ============================
%\usepackage[top=1.5cm,bottom=2cm,left=2.5cm,right=2.5cm,includehead]{geometry} % Marges personnalisées
%\usepackage{setspace}                       % Pour gérer l'interligne (\onehalfspacing, etc.)
% \usepackage{pdfpages}                     % Pour gérer l'insertion de PDF...
% \usepackage{afterpage}                    % Pour pouvoir forcer les sauts de page proprement : \afterpage{\clearpage} force l’insertion de toutes les figures en attente juste après cette page
% \usepackage[compact]{titlesec}            % Mise en forme des titres plus compacte
% \usepackage[table,xcdraw]{xcolor}         % Gérer les couleurs dans le document notamment dans les tableaux
% voir pour utiliser police / font perso (ex: Lato).                 % mise en page
%% Pour les commandes personnalisées

% --- Page blanche entre sections, sans numéro ---
\newcommand{\blankpage}{%
  \newpage
  \null
  \thispagestyle{empty}%
  \addtocounter{page}{-1}%
  \newpage
}

% --- Saut vertical court (optionnel) ---
\newcommand{\smallskipline}{\vspace{0.5\baselineskip}}

% --- Saut vertical moyen (optionnel) ---
\newcommand{\medskipline}{\vspace{1\baselineskip}}

% --- Saut vertical long (optionnel) ---
\newcommand{\bigskipline}{\vspace{2\baselineskip}}

% --- Ligne horizontale (utile en page de garde, etc.) ---
\newcommand{\HRule}{\rule{\linewidth}{0.4pt}}

% --- Commentaire en marge (léger) ---
\usepackage[textsize=footnotesize, color=orange!20]{todonotes}
\newcommand{\commentaire}[2][]{\todo[#1]{#2}}

% --- Paragraphe mis en valeur (style "new thought") ---
\newcommand{\newthought}[1]{\textbf{#1}\hspace{0.25em}}

% --- Environnement pour forcer une page droite non paginée ---
\newenvironment{rightpage}{%
  \cleardoublepage      % va à la page droite suivante
  \thispagestyle{empty} % pas de numéro
}{%
  \vfill                % pousse le contenu en haut
  \cleardoublepage      % termine sur une page droite vierge
}

% --- Mise en page article ---
               % commandes personnalisées

% ============================
% Langue et encodage
% ============================
\usepackage[english, french]{babel}         % Gestion multilingue, langue française par défaut (dernière) : typographie, noms automatiques, utiliser \selectlanguage{english} pour basculer en anglais et \begin{otherlanguage}{french} pour des inclusions fr suivi de \end{otherlanguage} puis de \selectlanguage{english} pour reprendre une partie en anglais.
\usepackage[utf8]{inputenc}                 % Encodage UTF-8 (utile en local)
\usepackage[T1]{fontenc}                    % Sortie correcte des accents
\usepackage{lmodern}                        % Police vectorielle moderne
% \usepackage{microtype}                    % Améliore la typographie (césure, justification)

% ============================
% Mise en page
% ============================
\usepackage[top=1.5cm,bottom=2cm,left=2.5cm,right=2.5cm,includehead]{geometry} % Marges personnalisées
\usepackage{setspace}                       % Pour gérer l'interligne (\onehalfspacing, etc.)
% \usepackage{pdfpages}                     % Pour gérer l'insertion de PDF...
% \usepackage{afterpage}                    % Pour pouvoir forcer les sauts de page proprement : \afterpage{\clearpage} force l’insertion de toutes les figures en attente juste après cette page
% \usepackage[compact]{titlesec}            % Mise en forme des titres plus compacte
% \usepackage[table,xcdraw]{xcolor}         % Gérer les couleurs dans le document notamment dans les tableaux
% voir pour utiliser police / font perso (ex: Lato).

% ============================
% Mathématiques
% ============================
\usepackage{amsmath, amssymb, amsfonts}     % Pour les formules et symboles mathématiques
\usepackage{siunitx}                        % Gestion des unités scientifiques (1.23 en et 1,23 fr) \SI{3.5}{mg/L} -> possibilité aussi de faire des unités personnalisées voir chatGPT au besoin

% ============================
% Images et figures
% ============================
\usepackage{graphicx}                       % Insertion d'images
\usepackage{float}                          % Placement forcé avec [H]
\usepackage[font=small,labelfont=bf,width=0.8\textwidth]{caption}   % personnaliser l’apparence des légendes (captions) des figures et tableaux
\usepackage{subcaption}                     % Sous-figures
% \usepackage{svg}                          % Pour inclure des fichiers SVG

% ============================
% Tableaux et couleurs
% ============================
\usepackage[table,xcdraw]{xcolor}           % Couleurs dans les tableaux
% \usepackage{booktabs}                     % Tableaux élégants (\toprule, etc.)
\usepackage{multirow}                       % Fusion de cellules dans les tableaux

% ============================
% Bibliographie
% ============================
\usepackage[backend=biber, style=apa,sorting=nyt, url=false]{biblatex} % Gestion biblio avancée
\addbibresource{references.bib}             % Fichier de bibliographie
% \addbibresource{biblio_manuelle.bib}      % Autre fichier possible -> utiliser betterbiblatex... ?
%\DefineBibliographyExtras{french}{\restorecommand\mkbibnamefamily} % Restaure un comportement cohérent en français pour la bibliographie dans la mise en forme des noms de famille

% ============================
% Hyperliens et citations
% ============================
\usepackage{hyperref}                       % Liens cliquables
\usepackage{csquotes}                       % Guillemets typographiques adaptés à la langue
% \usepackage{cleveref}                     % Références intelligentes (ex: \cref{fig:truc})

% ============================
% Glossaire et acronymes
% ============================
\usepackage[toc,acronym]{glossaries}
\makeglossaries
% \renewcommand*{\glsentryfmt}[1]{#1}       % Désactiver l'insertion automatique de la définition dans le texte
% ---------- MOTS ---------- %
\newglossaryentry{accession fine}{
    name={Accession fine}, 
    text={accession fine}, 
    description={Accession ou lignée de vigne donnant des grappes millerandées, c'est-à-dire avec des baies de tailles variables et un port lâche. Plus le degré de finesse est élevé, plus l'hétérogénéité est grande},
    plural={accessions fines}
}

\newglossaryentry{baguette}{
    name={Baguette}, 
    text={baguette}, 
    description={Terme viticole, notamment dans la taille en Guyot, désignant le bois de l'année précédente laissé sur le pied pour porter les rameaux fructifères de l'année suivante},
    plural={baguettes}
}

% ---------- ACRONYMES ---------- %
\newacronym{aba}{ABA}{Acide abscissique}
\newacronym{acp}{ACP}{Analyse en composantes principales}

% ---------- MODE D'EMPLOI ---------- %
% Une fois un mot ou acronyme déclaré avec \newglossaryentry ou \newacronym :
% Pour l’utiliser dans ton texte :
%\gls{nmr} → affiche NMR + explication complète la 1re fois (si configuré)

%\Gls{nmr} → idem, mais avec majuscule

%\glspl{nmr} → pluriel

%\acrshort{nmr} → affiche juste NMR

%\acrfull{nmr} → affiche acide abscissique (ABA) (forme longue + sigle)

%\acrlong{nmr} → juste la forme longue

% LaTeX se charge automatiquement d’écrire le mot complet la première fois, puis seulement l’abréviation ensuite.}

% ============================
% Gestion des annexes
% ============================
\usepackage{appendix}                       % Gestion propre des annexes

% ============================
% Divers / utile en rédaction
% ============================
% \usepackage{todonotes}                    % Pour ajouter des commentaires visibles dans le PDF

% ============================
% Commandes personnalisées
% ============================
% Création de page vierge entre les sections non comptées dans le doc
\newcommand\blankpage{%
  \null
  \thispagestyle{empty}%
  \addtocounter{page}{-1}%
  \newpage}

\begin{document}

\pagestyle{empty}

% Voir recommandations ED SVS !!!!!! puis adapter

\pagestyle{empty}

% Les logos en tête

\includegraphics[scale=1, height=1.7cm]{0_Liminaires/Images/UBX_logo.png}
\hfill

\includegraphics[scale=1, height=1.7cm]{0_Liminaires/Images/RF.png} %  remplacer par logo UMR (+ ISVV ?)
\hfill

\vspace{0.5cm}

% Bloc central
\begin{center}
\linespread{1.6}\selectfont % double interligne approximatif
\begin{Large}
THÈSE PRÉSENTÉE\\
POUR OBTENIR LE GRADE DE \\
{\LARGE \textbf{DOCTEUR\\DE L'UNIVERSITÉ DE BORDEAUX}}\\
\vspace{0.55cm}
ECOLE DOCTORALE SCIENCES DE LA VIE ET DE LA SANTÉ\\
{\normalsize Spécialité OENOLOGIE} \\
\vspace{0.55cm}
Par \textbf{Guillaume Séraphin BLANC LELEU} \\
\vspace{0.55cm}
{\Large Titre.}
\end{Large}

\vspace{0.55cm}

\linespread{1.05}\selectfont
{\normalsize
Sous la direction de : \textbf{Tristan RICHARD}\\
Responsable scientifique : \textbf{...}
}
\end{center}

\vfill

\begin{center}
{\large Soutenue le x novembre 2025 }\\    
\end{center}

\vfill

% Jury
Membres du jury :
\begin{table}[b]
\centering
\makebox[\textwidth]{%
\begin{tabular}{lllr}
Mme Prénom NOM & Grade & Université & Rapporteuse \\
Mme Prénom NOM & Grade & Université & Fonction \\
Mme Prénom NOM & Grade & Université & Fonction \\
Mme Prénom NOM & Grade & Université & Fonction \\
Mme Prénom NOM & Grade & Université & Fonction \\
\end{tabular}
}
\end{table}

\thispagestyle{empty}
\vfill
\newpage
\blankpage

% --- Résumé FR ---
\begin{small}
\linespread{1.05}\selectfont
\begin{center}
\textbf{Titre.}
\end{center}

\textbf{Résumé :} 
...

L.\newline
\\
\textbf{Mots-clés :} ... \\

\vspace{0.4cm}
\noindent\makebox[\linewidth]{\rule{\textwidth}{0.4pt}}

\vfill
\begin{center}
\textbf{Unité de recherche} \\
UMR 1366 OENO, 33140 Villenave-d'Ornon, France.
\end{center}
\newpage
\blankpage

% --- Abstract EN ---
\linespread{1.05}\selectfont
\begin{center}
\textbf{Title english.}
\end{center}

\textbf{Abstract:} 
...

\textbf{Keywords:} ...

\vspace{0.4cm}
\noindent\makebox[\linewidth]{\rule{\textwidth}{0.4pt}}

\vfill
\begin{center}
\textbf{Research unit} \\
UMR 1366 OENO, 33140 Villenave-d'Ornon, France.
\end{center}
\end{small}
\addcontentsline{toc}{section}{Citations}
\section*{Citations}
\textit{"Blablabla citation 1"}
\begin{flushright}
Auteur 1.
\end{flushright}

Alternative : \begin{quote} text \sourceatright{Auteur 1} \end{quote}
\begin{quotation} text \sourceatright{Auteur 1} \end{quotation}

\vspace{10pt}

\textit{"Blablabla citation 2"}
\begin{flushright}
Auteur 2.
\end{flushright}


\addcontentsline{toc}{section}{Remerciements}
\section*{Remerciements}

Fait toi plaisir quand tu en seras là c'est que tu auras déjà bien avancé ;)

Lorem ipsum dolor sit amet, consectetur adipiscing elit, sed do eiusmod tempor incididunt ut labore et dolore magna aliqua. Ut enim ad minim veniam, quis nostrud exercitation ullamco laboris nisi ut aliquip ex ea commodo consequat. Duis aute irure dolor in reprehenderit in voluptate velit esse cillum dolore eu fugiat nulla pariatur. Excepteur sint occaecat cupidatat non proident, sunt in culpa qui officia deserunt mollit anim id est laborum \parencite{leleu_cork_2025}.

Lorem ipsum dolor sit amet, consectetur adipiscing elit, sed do eiusmod tempor incididunt ut labore et dolore magna aliqua. Ut enim ad minim veniam, quis nostrud exercitation ullamco laboris nisi ut aliquip ex ea commodo consequat. Duis aute irure dolor in reprehenderit in voluptate velit esse cillum dolore eu fugiat nulla pariatur. Excepteur sint occaecat cupidatat non proident, sunt in culpa qui officia deserunt mollit anim id est laborum.

Lorem ipsum dolor sit amet, consectetur adipiscing elit, sed do eiusmod tempor incididunt ut labore et dolore magna aliqua. Ut enim ad minim veniam, quis nostrud exercitation ullamco laboris nisi ut aliquip ex ea commodo consequat. Duis aute irure dolor in reprehenderit in voluptate velit esse cillum dolore eu fugiat nulla pariatur. Excepteur sint occaecat cupidatat non proident, sunt in culpa qui officia deserunt mollit anim id est laborum \parencite{leleu_cork_2025}.

Lorem ipsum dolor sit amet, consectetur adipiscing elit, sed do eiusmod tempor incididunt ut labore et dolore magna aliqua. Ut enim ad minim veniam, quis nostrud exercitation ullamco laboris nisi ut aliquip ex ea commodo consequat. Duis aute irure dolor in reprehenderit in voluptate velit esse cillum dolore eu fugiat nulla pariatur. Excepteur sint occaecat cupidatat non proident, sunt in culpa qui officia deserunt mollit anim id est laborum.

Lorem ipsum dolor sit amet, consectetur adipiscing elit, sed do eiusmod tempor incididunt ut labore et dolore magna aliqua. Ut enim ad minim veniam, quis nostrud exercitation ullamco laboris nisi ut aliquip ex ea commodo consequat. Duis aute irure dolor in reprehenderit in voluptate velit esse cillum dolore eu fugiat nulla pariatur. Excepteur sint occaecat cupidatat non proident, sunt in culpa qui officia deserunt mollit anim id est laborum.

Lorem ipsum dolor sit amet, consectetur adipiscing elit, sed do eiusmod tempor incididunt ut labore et dolore magna aliqua. Ut enim ad minim veniam, quis nostrud exercitation ullamco laboris nisi ut aliquip ex ea commodo consequat. Duis aute irure dolor in reprehenderit in voluptate velit esse cillum dolore eu fugiat nulla pariatur. Excepteur sint occaecat cupidatat non proident, sunt in culpa qui officia deserunt mollit anim id est laborum.

Lorem ipsum dolor sit amet, consectetur adipiscing elit, sed do eiusmod tempor incididunt ut labore et dolore magna aliqua. Ut enim ad minim veniam, quis nostrud exercitation ullamco laboris nisi ut aliquip ex ea commodo consequat. Duis aute irure dolor in reprehenderit in voluptate velit esse cillum dolore eu fugiat nulla pariatur. Excepteur sint occaecat cupidatat non proident, sunt in culpa qui officia deserunt mollit anim id est laborum.
Lorem ipsum dolor sit amet, consectetur adipiscing elit, sed do eiusmod tempor incididunt ut labore et dolore magna aliqua. Ut enim ad minim veniam, quis nostrud exercitation ullamco laboris nisi ut aliquip ex ea commodo consequat. Duis aute irure dolor in reprehenderit in voluptate velit esse cillum dolore eu fugiat nulla pariatur. Excepteur sint occaecat cupidatat non proident, sunt in culpa qui officia deserunt mollit anim id est laborum.

% Maintenant déballe ton CV
\newpage
\addcontentsline{toc}{section}{Productions scientifiques}
\section*{Productions scientifiques}
\subsection*{Communications orales}
\begin{itemize}
    \item Workshop Modélisation 2024 - Toulouse (17/03/2024)
    \item ...
\end{itemize}

\subsection*{Posters scientifiques}

\begin{itemize}
    \item Journées Jeunes Chercheurs 2023 - Clermont-Ferrand (03/04/2023)
    \item ...
\end{itemize}

\vspace{10pt}

\subsection*{Activités annexes à la recherche}
\begin{itemize}
    \item Encadrement de stagiaires
    \item ADPI...
\end{itemize}

\pagestyle{plain}
\tableofcontents
% Mise en forme du glossaire contenu dans glossaire_definitions.tex

%\chapter*{Glossaire} % crée un chapitre non numéroté, il n’est pas ajouté automatiquement à la table des matières
%\addcontentsline{toc}{chapter}{Glossaire} % Cette commande ajoute manuellement une ligne dans la table des matières (ToC). toc = table of contents ; chapter = niveau hiérarchique de l’entrée (ici, un chapitre) ; Glossaire = le texte à afficher dans la ToC

% Glossaire et acronymes — à inclure avec % Mise en forme du glossaire contenu dans glossaire_definitions.tex

%\chapter*{Glossaire} % crée un chapitre non numéroté, il n’est pas ajouté automatiquement à la table des matières
%\addcontentsline{toc}{chapter}{Glossaire} % Cette commande ajoute manuellement une ligne dans la table des matières (ToC). toc = table of contents ; chapter = niveau hiérarchique de l’entrée (ici, un chapitre) ; Glossaire = le texte à afficher dans la ToC

% Glossaire et acronymes — à inclure avec % Mise en forme du glossaire contenu dans glossaire_definitions.tex

%\chapter*{Glossaire} % crée un chapitre non numéroté, il n’est pas ajouté automatiquement à la table des matières
%\addcontentsline{toc}{chapter}{Glossaire} % Cette commande ajoute manuellement une ligne dans la table des matières (ToC). toc = table of contents ; chapter = niveau hiérarchique de l’entrée (ici, un chapitre) ; Glossaire = le texte à afficher dans la ToC

% Glossaire et acronymes — à inclure avec \include{Liminaires/glossaire}

% Affiche la liste des acronymes
\printglossary[
  type=\acronymtype,
  title=Acronymes,
  toctitle=Acronymes % Alternative : Liste des acronymes, changera le titre uniquement dans la toc (table of content)
]

% Affiche le glossaire principal
\printglossary[
  title=Glossaire,
  toctitle=Glossaire % Glossaire des termes
]

% Utilisation à supprimer
Une fois un mot ou acronyme déclaré avec \newglossaryentry ou \newacronym :
Pour l’utiliser dans ton texte :
\gls{nmr} → affiche NMR + explication complète la 1re fois (si configuré)

\Gls{nmr} → idem, mais avec majuscule

\glspl{nmr} → pluriel

\acrshort{nmr} → affiche juste NMR

\acrfull{nmr} → affiche acide abscissique (ABA) (forme longue + sigle)

\acrlong{nmr} → juste la forme longue

LaTeX se charge automatiquement d’écrire le mot complet la première fois, puis seulement l’abréviation ensuite.

% Affiche la liste des acronymes
\printglossary[
  type=\acronymtype,
  title=Acronymes,
  toctitle=Acronymes % Alternative : Liste des acronymes, changera le titre uniquement dans la toc (table of content)
]

% Affiche le glossaire principal
\printglossary[
  title=Glossaire,
  toctitle=Glossaire % Glossaire des termes
]

% Utilisation à supprimer
Une fois un mot ou acronyme déclaré avec \newglossaryentry ou \newacronym :
Pour l’utiliser dans ton texte :
\gls{nmr} → affiche NMR + explication complète la 1re fois (si configuré)

\Gls{nmr} → idem, mais avec majuscule

\glspl{nmr} → pluriel

\acrshort{nmr} → affiche juste NMR

\acrfull{nmr} → affiche acide abscissique (ABA) (forme longue + sigle)

\acrlong{nmr} → juste la forme longue

LaTeX se charge automatiquement d’écrire le mot complet la première fois, puis seulement l’abréviation ensuite.

% Affiche la liste des acronymes
\printglossary[
  type=\acronymtype,
  title=Acronymes,
  toctitle=Acronymes % Alternative : Liste des acronymes, changera le titre uniquement dans la toc (table of content)
]

% Affiche le glossaire principal
\printglossary[
  title=Glossaire,
  toctitle=Glossaire % Glossaire des termes
]

% Utilisation à supprimer
Une fois un mot ou acronyme déclaré avec \newglossaryentry ou \newacronym :
Pour l’utiliser dans ton texte :
\gls{nmr} → affiche NMR + explication complète la 1re fois (si configuré)

\Gls{nmr} → idem, mais avec majuscule

\glspl{nmr} → pluriel

\acrshort{nmr} → affiche juste NMR

\acrfull{nmr} → affiche acide abscissique (ABA) (forme longue + sigle)

\acrlong{nmr} → juste la forme longue

LaTeX se charge automatiquement d’écrire le mot complet la première fois, puis seulement l’abréviation ensuite.
\addcontentsline{toc}{chapter}{Introduction générale}
\chapter*{Introduction générale}

...

Des analyses de répétabilité sont réalisées sur le spectromètre \textbf{500 MHz} avec des mesures \textbf{inter-day} en conditions de fidélité intermédiaire.
\textbf{Attention} à la comparaison 400 MHz / 500 MHz : certaines corrections pourraient manquer sur le \textbf{400 MHz}. Ne pas utiliser des données 400 et faire la démonstration uniquement sur la base du 500 MHz.

\subsubsection{Détermination profils d'exactitude des composés : incertitudes de mesure et LOD/LOQ}
Plusieurs méthodes d'estimation des limites de détection (LOD) et limites de quantification (LOQ) sont comparées, appliquées à un ou plusieurs composés -> Thèse

Présenter les profils d'exactitude desquels on peut déduire une incertitude relative ainsi que les LOQ (à 60 \% d'incertitude) et LOD. Proposer une estimation sur la base du S/N et des essais sur la répétabilité lorsque ce n'est pas possible.
\include{1_Chapitre_1_Etat_Art/chapitre_1_etat_art}
\include{2_Chapitre_2_Mat_et_Met/chapitre_2_materiel_et_methode}
\include{0_Liminaires/conclusion}

% \setcounter{biburllcpenalty}{7000} % A utiliser selon s'il y a des soucis de retour à la ligne dans la biblio (notamment sur les liens URL).
% \setcounter{biburlucpenalty}{8000}
% \renewcommand*{\labelnamepunct}{\newunitpunct\par} % Remplace le séparateur entre le nom de l’auteur et la suite de la référence par un saut de ligne sinon point ou virgule : plus aéré
% \renewbibmacro{in:}{\newline In:} % Personnalise le mot “In:” dans les références (pour les chapitres d’ouvrages, actes de congrès, etc.).

\newpage
\addcontentsline{toc}{chapter}{Bibliographie} % ajoute manuellement la Bibliographie dans la table des matières (utile si \printbibliography ne le fait pas automatiquement).
\printbibliography % imprime ta bibliographie complète.

\newpage
\addcontentsline{toc}{chapter}{Table des figures}
\listoffigures

\newpage
\addcontentsline{toc}{chapter}{Liste des tableaux}
\listoftables

% annexes
\chapter{Annexe Compléments sur les calculs}
Contenu...

\chapter{Annexe Script Python}
Contenu...

\end{document}

\section{Thèse sur article, dite aussi "thèse sur travaux" :}

Ce type de thèse présente les principaux résultats de la recherche sous forme d’articles publiés dans des revues scientifiques, ou sous forme d’articles acceptés pour publication, dans le domaine de recherche du doctorant.

 • Le doctorant peut être l’auteur unique d’un article ou son auteur principal, ce qui signifie :
 - qu’il précise, dans le cas d’articles co-signés, sa contribution qui doit être significative,
 - que les articles correspondent à une contribution scientifique originale,

 • Les articles qui constituent le corps du document doivent être :
 - précédés dans le mémoire de thèse, d’une présentation substantielle,
 - suivis d’une synthèse substantielle qui comprend une discussion générale des résultats, des conclusions et une bibliographie.

Les parties nécessaires à la compréhension générale du mémoire mais qui ne peuvent pas paraître dans les journaux scientifiques doivent être fournies (techniques ou méthodes d’analyse spécifiques, données supplémentaires, etc.).

\section{Redaction du manuscrit :}
La rédaction du mémoire est l’une des formations apportées par le doctorat et constitue une part essentielle du travail de thèse.
Le mémoire doit montrer les capacités du candidat à présenter l’objectif de son étude, faire l’état de l’art sur la question, situer son travail dans le contexte international, exposer la méthodologie, développer les différents aspects de son travail. Ses capacités à rédiger seront utiles au docteur au cours de sa vie professionnelle, qu’elle se réalise dans le milieu académique ou en entreprise.

\subparagraph{Contenu :}

La forme traditionnelle du mémoire est constituée de différents chapitres que le doctorant s’attachera à articuler entre eux. Il présentera une synthèse critique des résultats obtenus avant de proposer des conclusions et des perspectives.Le mémoire pourra intégrer des articles publiés ou acceptés pour publication dans des revues reconnues dans le domaine de recherche du doctorant à condition :
• qu’il en soit l’artisan principal et qu’il précise, dans le cas d’articles co-signés, sa contribution qui doit être significative,
• que les articles correspondent à des contributions scientifiques originales,
• qu’ils soient précédés, dans le mémoire de thèse, d’une présentation substantielle montrant comment ils s’intègrent dans le travail de thèse,
• que les parties nécessaires à la compréhension générale du mémoire mais qui ne peuvent pas paraître dans les journaux scientifiques soient fournies (techniques ou méthodes d’analyse spécifiques, données supplémentaires, etc.).

Dans le cas où de larges passages du manuscrit seraient constitués d'articles, il est demandé un travail de rédaction d'au minimum 40 pages introduisant le sujet, présentant l'état de l'art et toute partie ne figurant pas dans les articles mais étant nécessaire à la compréhension de l’ensemble, liant les articles entre eux et concluant sur les objectifs atteints et les perspectives.
Des publications ne sauraient, en aucun cas, représenter à elles seules la thèse elle-même.

Le rôle des membres du jury sera précisé après soutenance dans la version définitive du manuscrit de thèse.

\subparagraph{L’usage des langues pour la rédaction du mémoire et la soutenance de thèse :}

La langue de rédaction est normalement le français.

Le mémoire contient obligatoirement un résumé rédigé en français et en anglais.

La langue de soutenance est normalement le français, sauf pour les étudiants non francophones, ou ceux dont le travail s'est développé dans un cadre international, auquel cas l'usage de la langue anglaise est toléré.
La discussion pourra se dérouler en anglais si des membres du jury le souhaitent.
Les rapporteurs et le président du jury devront rédiger leurs rapports en suivant ce qui est prévu dans la convention de thèse si elle existe, en français ou en anglais.

 