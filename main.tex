% ============================
% Classe du document
% ============================
\documentclass[a4paper,12pt]{report}        % 'report' est bien adapté aux thèses

% ============================
% Langue et encodage
% ============================
\usepackage[english, french]{babel}         % Gestion multilingue, langue française par défaut (dernière) : typographie, noms automatiques, utiliser \selectlanguage{english} pour basculer en anglais et \begin{otherlanguage}{french} pour des inclusions fr suivi de \end{otherlanguage} puis de \selectlanguage{english} pour reprendre une partie en anglais.
\usepackage[utf8]{inputenc}                 % Encodage UTF-8 (utile en local)
\usepackage[T1]{fontenc}                    % Sortie correcte des accents
\usepackage{lmodern}                        % Police vectorielle moderne
% \usepackage{microtype}                    % Améliore la typographie (césure, justification)

% ============================
% Mise en page
% ============================
\usepackage[top=1.5cm,bottom=2cm,left=2.5cm,right=2.5cm,includehead]{geometry} % Marges personnalisées
\usepackage{setspace}                       % Pour gérer l'interligne (\onehalfspacing, etc.)
% \usepackage{pdfpages}                     % Pour gérer l'insertion de PDF...
% \usepackage{afterpage}                    % Pour pouvoir forcer les sauts de page proprement : \afterpage{\clearpage} force l’insertion de toutes les figures en attente juste après cette page
% \usepackage[compact]{titlesec}            % Mise en forme des titres plus compacte
% \usepackage[table,xcdraw]{xcolor}         % Gérer les couleurs dans le document notamment dans les tableaux

% ============================
% Mathématiques
% ============================
\usepackage{amsmath, amssymb, amsfonts}     % Pour les formules et symboles mathématiques
\usepackage{siunitx}                        % Gestion des unités scientifiques (1.23 en et 1,23 fr) \SI{3.5}{mg/L} -> possibilité aussi de faire des unités personnalisées voir chatGPT au besoin

% ============================
% Images et figures
% ============================
\usepackage{graphicx}                       % Insertion d'images
\usepackage{float}                          % Placement forcé avec [H]
\usepackage[font=small,labelfont=bf,width=0.8\textwidth]{caption}   % personnaliser l’apparence des légendes (captions) des figures et tableaux
\usepackage{subcaption}                     % Sous-figures
% \usepackage{svg}                          % Pour inclure des fichiers SVG

% ============================
% Tableaux et couleurs
% ============================
\usepackage[table,xcdraw]{xcolor}           % Couleurs dans les tableaux
% \usepackage{booktabs}                     % Tableaux élégants (\toprule, etc.)
\usepackage{multirow}                       % Fusion de cellules dans les tableaux

% ============================
% Bibliographie
% ============================
\usepackage[backend=biber, style=apa,sorting=nyt, url=false, Extra=false]{biblatex} % Gestion biblio avancée
\addbibresource{references.bib}             % Fichier de bibliographie
% \addbibresource{biblio_manuelle.bib}      % Autre fichier possible
%\DefineBibliographyExtras{french}{\restorecommand\mkbibnamefamily} % Restaure un comportement cohérent en français pour la bibliographie dans la mise en forme des noms de famille

% ============================
% Hyperliens et citations
% ============================
\usepackage{hyperref}                       % Liens cliquables
\usepackage{csquotes}                       % Guillemets typographiques adaptés à la langue
% \usepackage{cleveref}                     % Références intelligentes (ex: \cref{fig:truc})

% ============================
% Glossaire et acronymes
% ============================
\usepackage[toc,acronym]{glossaries}
\makeglossaries
% \renewcommand*{\glsentryfmt}[1]{#1}       % Désactiver l'insertion automatique de la définition dans le texte
% ---------- MOTS ---------- %
\newglossaryentry{accession fine}{
    name={Accession fine}, 
    text={accession fine}, 
    description={Accession ou lignée de vigne donnant des grappes millerandées, c'est-à-dire avec des baies de tailles variables et un port lâche. Plus le degré de finesse est élevé, plus l'hétérogénéité est grande},
    plural={accessions fines}
}

\newglossaryentry{baguette}{
    name={Baguette}, 
    text={baguette}, 
    description={Terme viticole, notamment dans la taille en Guyot, désignant le bois de l'année précédente laissé sur le pied pour porter les rameaux fructifères de l'année suivante},
    plural={baguettes}
}

% ---------- ACRONYMES ---------- %
\newacronym{aba}{ABA}{Acide abscissique}
\newacronym{acp}{ACP}{Analyse en composantes principales}

% ---------- MODE D'EMPLOI ---------- %
% Une fois un mot ou acronyme déclaré avec \newglossaryentry ou \newacronym :
% Pour l’utiliser dans ton texte :
%\gls{nmr} → affiche NMR + explication complète la 1re fois (si configuré)

%\Gls{nmr} → idem, mais avec majuscule

%\glspl{nmr} → pluriel

%\acrshort{nmr} → affiche juste NMR

%\acrfull{nmr} → affiche acide abscissique (ABA) (forme longue + sigle)

%\acrlong{nmr} → juste la forme longue

% LaTeX se charge automatiquement d’écrire le mot complet la première fois, puis seulement l’abréviation ensuite.}

% ============================
% Gestion des annexes
% ============================
\usepackage{appendix}                       % Gestion propre des annexes

% ============================
% Divers / utile en rédaction
% ============================
% \usepackage{todonotes}                    % Pour ajouter des commentaires visibles dans le PDF

% ============================
% Commandes personnalisées
% ============================
% Création de page vierge entre les sections non comptées dans le doc
\newcommand\blankpage{%
  \null
  \thispagestyle{empty}%
  \addtocounter{page}{-1}%
  \newpage}

\begin{document}

\pagestyle{empty}

\include{0_Liminaires/page_garde_abstract}
\addcontentsline{toc}{section}{Citations}
\section*{Citations}
\textit{"Blablabla citation 1"}
\begin{flushright}
Auteur 1.
\end{flushright}

Alternative : \begin{quote} text \sourceatright{Auteur 1} \end{quote}
\begin{quotation} text \sourceatright{Auteur 1} \end{quotation}

\vspace{10pt}

\textit{"Blablabla citation 2"}
\begin{flushright}
Auteur 2.
\end{flushright}


\addcontentsline{toc}{section}{Remerciements}
\section*{Remerciements}

Fait toi plaisir quand tu en seras là c'est que tu auras déjà bien avancé ;)

Lorem ipsum dolor sit amet, consectetur adipiscing elit, sed do eiusmod tempor incididunt ut labore et dolore magna aliqua. Ut enim ad minim veniam, quis nostrud exercitation ullamco laboris nisi ut aliquip ex ea commodo consequat. Duis aute irure dolor in reprehenderit in voluptate velit esse cillum dolore eu fugiat nulla pariatur. Excepteur sint occaecat cupidatat non proident, sunt in culpa qui officia deserunt mollit anim id est laborum \parencite{leleu_cork_2025}.

Lorem ipsum dolor sit amet, consectetur adipiscing elit, sed do eiusmod tempor incididunt ut labore et dolore magna aliqua. Ut enim ad minim veniam, quis nostrud exercitation ullamco laboris nisi ut aliquip ex ea commodo consequat. Duis aute irure dolor in reprehenderit in voluptate velit esse cillum dolore eu fugiat nulla pariatur. Excepteur sint occaecat cupidatat non proident, sunt in culpa qui officia deserunt mollit anim id est laborum.

Lorem ipsum dolor sit amet, consectetur adipiscing elit, sed do eiusmod tempor incididunt ut labore et dolore magna aliqua. Ut enim ad minim veniam, quis nostrud exercitation ullamco laboris nisi ut aliquip ex ea commodo consequat. Duis aute irure dolor in reprehenderit in voluptate velit esse cillum dolore eu fugiat nulla pariatur. Excepteur sint occaecat cupidatat non proident, sunt in culpa qui officia deserunt mollit anim id est laborum \parencite{leleu_cork_2025}.

Lorem ipsum dolor sit amet, consectetur adipiscing elit, sed do eiusmod tempor incididunt ut labore et dolore magna aliqua. Ut enim ad minim veniam, quis nostrud exercitation ullamco laboris nisi ut aliquip ex ea commodo consequat. Duis aute irure dolor in reprehenderit in voluptate velit esse cillum dolore eu fugiat nulla pariatur. Excepteur sint occaecat cupidatat non proident, sunt in culpa qui officia deserunt mollit anim id est laborum.

Lorem ipsum dolor sit amet, consectetur adipiscing elit, sed do eiusmod tempor incididunt ut labore et dolore magna aliqua. Ut enim ad minim veniam, quis nostrud exercitation ullamco laboris nisi ut aliquip ex ea commodo consequat. Duis aute irure dolor in reprehenderit in voluptate velit esse cillum dolore eu fugiat nulla pariatur. Excepteur sint occaecat cupidatat non proident, sunt in culpa qui officia deserunt mollit anim id est laborum.

Lorem ipsum dolor sit amet, consectetur adipiscing elit, sed do eiusmod tempor incididunt ut labore et dolore magna aliqua. Ut enim ad minim veniam, quis nostrud exercitation ullamco laboris nisi ut aliquip ex ea commodo consequat. Duis aute irure dolor in reprehenderit in voluptate velit esse cillum dolore eu fugiat nulla pariatur. Excepteur sint occaecat cupidatat non proident, sunt in culpa qui officia deserunt mollit anim id est laborum.

Lorem ipsum dolor sit amet, consectetur adipiscing elit, sed do eiusmod tempor incididunt ut labore et dolore magna aliqua. Ut enim ad minim veniam, quis nostrud exercitation ullamco laboris nisi ut aliquip ex ea commodo consequat. Duis aute irure dolor in reprehenderit in voluptate velit esse cillum dolore eu fugiat nulla pariatur. Excepteur sint occaecat cupidatat non proident, sunt in culpa qui officia deserunt mollit anim id est laborum.
Lorem ipsum dolor sit amet, consectetur adipiscing elit, sed do eiusmod tempor incididunt ut labore et dolore magna aliqua. Ut enim ad minim veniam, quis nostrud exercitation ullamco laboris nisi ut aliquip ex ea commodo consequat. Duis aute irure dolor in reprehenderit in voluptate velit esse cillum dolore eu fugiat nulla pariatur. Excepteur sint occaecat cupidatat non proident, sunt in culpa qui officia deserunt mollit anim id est laborum.

% Maintenant déballe ton CV
\newpage
\addcontentsline{toc}{section}{Productions scientifiques}
\section*{Productions scientifiques}
\subsection*{Communications orales}
\begin{itemize}
    \item Workshop Modélisation 2024 - Toulouse (17/03/2024)
    \item ...
\end{itemize}

\subsection*{Posters scientifiques}

\begin{itemize}
    \item Journées Jeunes Chercheurs 2023 - Clermont-Ferrand (03/04/2023)
    \item ...
\end{itemize}

\vspace{10pt}

\subsection*{Activités annexes à la recherche}
\begin{itemize}
    \item Encadrement de stagiaires
    \item ADPI...
\end{itemize}

\pagestyle{plain}
\tableofcontents
% Mise en forme du glossaire contenu dans glossaire_definitions.tex

%\chapter*{Glossaire} % crée un chapitre non numéroté, il n’est pas ajouté automatiquement à la table des matières
%\addcontentsline{toc}{chapter}{Glossaire} % Cette commande ajoute manuellement une ligne dans la table des matières (ToC). toc = table of contents ; chapter = niveau hiérarchique de l’entrée (ici, un chapitre) ; Glossaire = le texte à afficher dans la ToC

% Glossaire et acronymes — à inclure avec % Mise en forme du glossaire contenu dans glossaire_definitions.tex

%\chapter*{Glossaire} % crée un chapitre non numéroté, il n’est pas ajouté automatiquement à la table des matières
%\addcontentsline{toc}{chapter}{Glossaire} % Cette commande ajoute manuellement une ligne dans la table des matières (ToC). toc = table of contents ; chapter = niveau hiérarchique de l’entrée (ici, un chapitre) ; Glossaire = le texte à afficher dans la ToC

% Glossaire et acronymes — à inclure avec % Mise en forme du glossaire contenu dans glossaire_definitions.tex

%\chapter*{Glossaire} % crée un chapitre non numéroté, il n’est pas ajouté automatiquement à la table des matières
%\addcontentsline{toc}{chapter}{Glossaire} % Cette commande ajoute manuellement une ligne dans la table des matières (ToC). toc = table of contents ; chapter = niveau hiérarchique de l’entrée (ici, un chapitre) ; Glossaire = le texte à afficher dans la ToC

% Glossaire et acronymes — à inclure avec \include{Liminaires/glossaire}

% Affiche la liste des acronymes
\printglossary[
  type=\acronymtype,
  title=Acronymes,
  toctitle=Acronymes % Alternative : Liste des acronymes, changera le titre uniquement dans la toc (table of content)
]

% Affiche le glossaire principal
\printglossary[
  title=Glossaire,
  toctitle=Glossaire % Glossaire des termes
]

% Utilisation à supprimer
Une fois un mot ou acronyme déclaré avec \newglossaryentry ou \newacronym :
Pour l’utiliser dans ton texte :
\gls{nmr} → affiche NMR + explication complète la 1re fois (si configuré)

\Gls{nmr} → idem, mais avec majuscule

\glspl{nmr} → pluriel

\acrshort{nmr} → affiche juste NMR

\acrfull{nmr} → affiche acide abscissique (ABA) (forme longue + sigle)

\acrlong{nmr} → juste la forme longue

LaTeX se charge automatiquement d’écrire le mot complet la première fois, puis seulement l’abréviation ensuite.

% Affiche la liste des acronymes
\printglossary[
  type=\acronymtype,
  title=Acronymes,
  toctitle=Acronymes % Alternative : Liste des acronymes, changera le titre uniquement dans la toc (table of content)
]

% Affiche le glossaire principal
\printglossary[
  title=Glossaire,
  toctitle=Glossaire % Glossaire des termes
]

% Utilisation à supprimer
Une fois un mot ou acronyme déclaré avec \newglossaryentry ou \newacronym :
Pour l’utiliser dans ton texte :
\gls{nmr} → affiche NMR + explication complète la 1re fois (si configuré)

\Gls{nmr} → idem, mais avec majuscule

\glspl{nmr} → pluriel

\acrshort{nmr} → affiche juste NMR

\acrfull{nmr} → affiche acide abscissique (ABA) (forme longue + sigle)

\acrlong{nmr} → juste la forme longue

LaTeX se charge automatiquement d’écrire le mot complet la première fois, puis seulement l’abréviation ensuite.

% Affiche la liste des acronymes
\printglossary[
  type=\acronymtype,
  title=Acronymes,
  toctitle=Acronymes % Alternative : Liste des acronymes, changera le titre uniquement dans la toc (table of content)
]

% Affiche le glossaire principal
\printglossary[
  title=Glossaire,
  toctitle=Glossaire % Glossaire des termes
]

% Utilisation à supprimer
Une fois un mot ou acronyme déclaré avec \newglossaryentry ou \newacronym :
Pour l’utiliser dans ton texte :
\gls{nmr} → affiche NMR + explication complète la 1re fois (si configuré)

\Gls{nmr} → idem, mais avec majuscule

\glspl{nmr} → pluriel

\acrshort{nmr} → affiche juste NMR

\acrfull{nmr} → affiche acide abscissique (ABA) (forme longue + sigle)

\acrlong{nmr} → juste la forme longue

LaTeX se charge automatiquement d’écrire le mot complet la première fois, puis seulement l’abréviation ensuite.
\addcontentsline{toc}{chapter}{Introduction générale}
\chapter*{Introduction générale}

...

Des analyses de répétabilité sont réalisées sur le spectromètre \textbf{500 MHz} avec des mesures \textbf{inter-day} en conditions de fidélité intermédiaire.
\textbf{Attention} à la comparaison 400 MHz / 500 MHz : certaines corrections pourraient manquer sur le \textbf{400 MHz}. Ne pas utiliser des données 400 et faire la démonstration uniquement sur la base du 500 MHz.

\subsubsection{Détermination profils d'exactitude des composés : incertitudes de mesure et LOD/LOQ}
Plusieurs méthodes d'estimation des limites de détection (LOD) et limites de quantification (LOQ) sont comparées, appliquées à un ou plusieurs composés -> Thèse

Présenter les profils d'exactitude desquels on peut déduire une incertitude relative ainsi que les LOQ (à 60 \% d'incertitude) et LOD. Proposer une estimation sur la base du S/N et des essais sur la répétabilité lorsque ce n'est pas possible.
\include{1_Chapitre_1_Etat_Art/chapitre_1_etat_art}
\include{2_Chapitre_2_Mat_et_Met/chapitre_2_materiel_et_methode}
\include{0_Liminaires/conclusion}

% \setcounter{biburllcpenalty}{7000} % A utiliser selon s'il y a des soucis de retour à la ligne dans la biblio (notamment sur les liens URL).
% \setcounter{biburlucpenalty}{8000}
% \renewcommand*{\labelnamepunct}{\newunitpunct\par} % Remplace le séparateur entre le nom de l’auteur et la suite de la référence par un saut de ligne sinon point ou virgule : plus aéré
% \renewbibmacro{in:}{\newline In:} % Personnalise le mot “In:” dans les références (pour les chapitres d’ouvrages, actes de congrès, etc.).

\newpage
\addcontentsline{toc}{chapter}{Bibliographie} % ajoute manuellement la Bibliographie dans la table des matières (utile si \printbibliography ne le fait pas automatiquement).
\printbibliography % imprime ta bibliographie complète.

\newpage
\addcontentsline{toc}{chapter}{Table des figures}
\listoffigures

\newpage
\addcontentsline{toc}{chapter}{Liste des tableaux}
\listoftables

% annexes
\chapter{Annexe Compléments sur les calculs}
Contenu...

\chapter{Annexe Script Python}
Contenu...

\end{document}
